\documentclass{article}
\usepackage[utf8]{inputenc}
\usepackage[T1]{fontenc}
\usepackage{amsmath, amsfonts, amscd, amsthm}
\usepackage[left=1in, right=1in, top=1in, bottom=1in]{geometry}

\newtheorem{theorem}{Theorem}[section]
\newtheorem{exercise}[theorem]{Exercise}
\newtheorem{question}[theorem]{Question}

\newenvironment{definition}[1][Definition]
{
	\begin{trivlist}
		\item[\hskip \labelsep {\bfseries #1}]
	}
	{
	\end{trivlist}
}



\title{HW Solution}
%Put your name here.
\author{First Last}
%Check the due date.
\date{01/25/2017}

\begin{document}
\maketitle
%Set double spacing
\linespread{2}
%Counters to manipulate theorem numbers
\setcounter{section}{1}
\setcounter{theorem}{0}

\begin{theorem}
Let a, b, and c be integers. If \(a|b\) and \(a|c\), then \(a|(b+c)\).
%Note: The pair, \(  \), puts everything inside into the 'in-line math type environment'. For equations on their own line, use \align{} or \align*{} (for no numbering).
\end{theorem}

\begin{proof}
Here is where you write your proof.
\end{proof}

\begin{theorem}
...
\end{theorem}

\begin{proof}
...
\end{proof}

\begin{theorem}
...
\end{theorem}

\begin{proof}
...
\end{proof}

Here are some examples of how to use other environments, modify standard ones, and how to use counters.
\setcounter{section}{3}
\setcounter{theorem}{4}
\begin{theorem}[The Squeeze Theorem]
	...
\end{theorem}

\begin{proof}
	...
\end{proof}

\setcounter{section}{2}
\setcounter{theorem}{7}
\begin{question}
	...
\end{question}

\begin{proof}[Answer]
	...
\end{proof}


\setcounter{section}{6}
\setcounter{theorem}{1}
\begin{exercise}
	...
\end{exercise}

\begin{proof}[Solution]
	...
\end{proof}

\begin{definition}
	...
\end{definition}

\begin{definition}[Integers]
	...
\end{definition}

\end{document}